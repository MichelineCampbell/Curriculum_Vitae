%% start of file `template.tex'.
%% Copyright 2006-2015 Xavier Danaux (xdanaux@gmail.com).
%
% Adapted to be an Rmarkdown template by Mitchell O'Hara-Wild
% 8 February 2019
%
% This work may be distributed and/or modified under the
% conditions of the LaTeX Project Public License version 1.3c,
% available at http://www.latex-project.org/lppl/.


\documentclass[11pt,a4paper,]{moderncv}

% moderncv themes
\moderncvstyle{classic}                             % style options are 'casual' (default), 'classic', 'banking', 'oldstyle' and 'fancy'

\definecolor{color0}{rgb}{0,0,0}% black
\definecolor{color1}{HTML}{125656}% custom
\definecolor{color2}{rgb}{0.45,0.45,0.45}% dark grey

\usepackage[scaled=0.86]{DejaVuSansMono}

\providecommand{\tightlist}{%
	\setlength{\itemsep}{0pt}\setlength{\parskip}{0pt}}
\def\donothing#1{#1}
\def\emaillink#1{#1}

%\nopagenumbers{}                                  % uncomment to suppress automatic page numbering for CVs longer than one page

% character encoding
%\usepackage[utf8]{inputenc}                       % if you are not using xelatex ou lualatex, replace by the encoding you are using
%\usepackage{CJKutf8}                              % if you need to use CJK to typeset your resume in Chinese, Japanese or Korean

% adjust the page margins
\usepackage[scale=0.75,footskip=60pt]{geometry}
%\setlength{\hintscolumnwidth}{3cm}                % if you want to change the width of the column with the dates
%\setlength{\makecvheadnamewidth}{10cm}            % for the 'classic' style, if you want to force the width allocated to your name and avoid line breaks. be careful though, the length is normally calculated to avoid any overlap with your personal info; use this at your own typographical risks...



% personal data
\name{}{Dr.~Micheline Campbell}
\title{Research Associate, UNSW Sydney}
\address{Sydney, Australia}{}{}
\phone[mobile]{+61 411 282 657} % Phone number
\email{\donothing{\href{mailto:michelineleecampbell@gmail.com}{\nolinkurl{michelineleecampbell@gmail.com}}}}
 % Personal website

\social[twitter]{michcampbell}
\social[github]{MichelineCampbell}
% \extrainfo{additional information}                 % optional, remove / comment the line if not wanted




% Pandoc CSL macros
\newlength{\cslhangindent}
\setlength{\cslhangindent}{1.5em}
\newlength{\csllabelwidth}
\setlength{\csllabelwidth}{3em}
\newenvironment{CSLReferences}[3] % #1 hanging-ident, #2 entry spacing
 {% don't indent paragraphs
  \setlength{\parindent}{0pt}
  % turn on hanging indent if param 1 is 1
  \ifodd #1 \everypar{\setlength{\hangindent}{\cslhangindent}}\ignorespaces\fi
  % set entry spacing
  \ifnum #2 > 0
  \setlength{\parskip}{#2\baselineskip}
  \fi
 }%
 {}
\usepackage{calc}
\newcommand{\CSLBlock}[1]{#1\hfill\break}
\newcommand{\CSLLeftMargin}[1]{\parbox[t]{\csllabelwidth}{#1}}
\newcommand{\CSLRightInline}[1]{\parbox[t]{\linewidth - \csllabelwidth}{#1}}
\newcommand{\CSLIndent}[1]{\hspace{\cslhangindent}#1}

%----------------------------------------------------------------------------------
%            content
%----------------------------------------------------------------------------------
\begin{document}
%\begin{CJK*}{UTF8}{gbsn}                          % to typeset your resume in Chinese using CJK
%-----       resume       ---------------------------------------------------------
\makecvtitle



\hypertarget{professional-summary}{%
\section{Professional Summary}\label{professional-summary}}

I have a PhD in speleothem palaeo-climatology and am currently employed
as a research associate at the University of New South Wales,
contributing to the Australian Research Council Discovery Project
``Reconstructing Australia's fire history from cave stalagmites.'' I
have strong data and field science skills and have developed a research
skillset applicable to a broad range of topics within the environmental
sciences. My PhD research focused on stalagmite-based climate proxy
records and associated karst geomorphology and groundwater hydrology,
and since then I have worked in the surface hydrology and
palaeoclimate-water security nexus. I am hoping to continue my career in
physical geography by securing a position where I can apply skills I
have developed in a meaningful way and contribute to the development of
the next generation of geographers.

\hypertarget{education}{%
\section{Education}\label{education}}

\nopagebreak
    \cventry{Aug 2014 -
Jul 2019}{PhD (Geography)}{School of Agriculture and Environment, The University of Western Australia}{Perth, Western Australia, Australia}{}{\begin{itemize}%
\item  \textbf{Thesis: Speleothem-based palaeo-climate research: Methodology, applications, and insight from the Snowy Mountains, southeast Australia}%
\item Supervisors: JN. Callow, H. McGowan, G.McGrath, H. Wong%
\end{itemize}}
    \cventry{Jul 2013 -
Jun 2014}{BA (Hons, 1st Class) (Geography)}{School of Geography, Planning, and Environmental Management, The University of Queensland}{Brisbane, QLD, Australia}{}{\begin{itemize}%
\item  \textbf{Thesis: The Drip Hydrology of a Speleothem from the Yarrangobilly Caves, NSW, Australia}%
\item Supervisors: H. Mcgowan, JN. Callow%
\end{itemize}}\nopagebreak
    \cventry{Feb 2010 -
Dec 2012}{BA (Geography and Spanish)}{The University of Queensland}{Brisbane, QLD, Australia}{}{\empty}

\hypertarget{experience}{%
\section{Experience}\label{experience}}

\hypertarget{research}{%
\subsection{Research}\label{research}}

\nopagebreak
    \cventry{Jun 2021 -
present}{Research Associate}{School of Biological, Earth and Environmental Sciences, University of New South Wales}{Sydney, NSW, Australia}{}{\begin{itemize}%
\item Contributing to the ARC Discovery Project ``Reconstructing Australia’s fire history from cave stalagmites''%
\end{itemize}}
    \cventry{Jul 2020 -
Jun 2021}{Postdoctoral Researcher}{School of Geography, University College Dublin}{Dublin/Brisbane, QLD, Ireland/Australia}{}{\begin{itemize}%
\item Contributing to the project 'Using Palaeo-climate Proxies for Water Security Planning'%
\item Undertook the majority of data science tasks for the development of a new database of Australian climate proxy data%
\end{itemize}}
    \cventry{Feb 2020 -
Jul 2020}{Scientist}{Land and Water Science Unit, QLD Dept. of Natural Resources, Mining and Energy}{Rockhampton, QLD, Australia}{}{\begin{itemize}%
\item Analysed sugar cane trash trial data in relation to nutrient and sediment  transport.%
\item Found that a significant proportion of the variation of sediment transported in cane paddock runoff can be described by soil type and cane trash cover.%
\item This was a short term 5-month contract%
\end{itemize}}
    \cventry{Aug 2019 -
Dec 2019}{Research Assistant}{School of Agriculture and Environment, University of Western Australia}{Perth, WA, Australia}{}{\begin{itemize}%
\item Miscellaneous data analysis tasks%
\end{itemize}}
    \cventry{Aug 2014 -
Jul 2019}{PhD Researcher}{School of Agriculture and Environment, University of Western Australia}{Perth, WA, Australia}{}{\begin{itemize}%
\item Developed two new palaeo-climate proxy records for south east Australia%
\item Showed that the climate of the LIG had similar significant periodicities to present, suggesting our long term climate patterns (e.g. PDO) will continue as temperatures increase with climate change.%
\item Used social network analysis to advance our understanding of methodologies applied in the speleothem palaeo-climate proxy community and the development of the field over time.%
\item Demonstrated that LA-ICP-MS and LA-ICP-AES of trace elements in speleothem calcite produce comparable results%
\end{itemize}}

\hypertarget{teaching}{%
\subsection{Teaching}\label{teaching}}

\nopagebreak
    \cventry{2018}{Tutor: Geographies of a Global City}{School of Agriculture and Enviroment, The University of Western Australia}{Perth, WA, Australia}{}{\empty}
    \cventry{2018}{Course Co-Coordinator and Lecturer: Catchment and River Processes}{School of Agriculture and Enviroment, The University of Western Australia}{Perth, WA, Australia}{}{\empty}
    \cventry{2014--2018}{Tutor: Catchment and River Processes}{School of Agriculture and Enviroment, The University of Western Australia}{Perth, WA, Australia}{}{\empty}

Teaching contributions include the delivery and creation of content for
undergraduate and masters level units, both in the classroom/laboratory
and in the field. Catchment and River Processes, which I taught into
since its inception at UWA, consistently achieved high student ratings,
including in 2018 when it received a score of 3.7/4. This placed it in
the 70\textsuperscript{th} percentile of UWA courses.

\newpage

\hypertarget{funding-and-scholarships}{%
\section{Funding and Scholarships}\label{funding-and-scholarships}}

\nopagebreak
    \cventry{2021}{Australian Institute of Nuclear Science and Engineering Early Career Researcher Grant}{}{}{}{\begin{itemize}%
\item \$10 000%
\end{itemize}}
    \cventry{2021}{University of New South Wales UNSW Science Covid-19 Strategic Support Grant}{}{}{}{\begin{itemize}%
\item \$4 000%
\end{itemize}}
    \cventry{2018}{Australian Institute of Nuclear Science and Engineering Travel grant}{}{}{}{\begin{itemize}%
\item \$1 000%
\end{itemize}}
    \cventry{2017}{Australian and New Zealand Geomorphological Group Travel grant}{}{}{}{\begin{itemize}%
\item \$300%
\end{itemize}}
    \cventry{2016}{University of Western Australia Travel grant}{}{}{}{\begin{itemize}%
\item \$1 850%
\end{itemize}}
    \cventry{2015--2018}{Australian Institute of Nuclear Science and Engineering Postgraduate Ressarch Award Stipend}{}{}{}{\begin{itemize}%
\item \$7 500pa%
\end{itemize}}
    \cventry{2015--2018}{Australian Institute of Nuclear Science and Engineering Postgraduate Ressarch Award Analytical Budget}{}{}{}{\begin{itemize}%
\item \$10 000pa%
\end{itemize}}
    \cventry{2014--2018}{Australian Government Research Training Program Stipend}{}{}{}{\begin{itemize}%
\item \$26 000pa%
\end{itemize}}
    \cventry{2014--2015}{Snowy Hydro Ltd. Stipend}{}{}{}{\begin{itemize}%
\item \$5 000pa%
\end{itemize}}
    \cventry{2014}{University of Western Australia Stipend}{}{}{}{\begin{itemize}%
\item \$3 000%
\end{itemize}}

\hypertarget{publications}{%
\section{Publications}\label{publications}}

\hypertarget{published}{%
\subsection{Published}\label{published}}

\hypertarget{bibliography}{}
\leavevmode\hypertarget{ref-croke_palaeoclimate_2021}{}%
\CSLLeftMargin{1. }
\CSLRightInline{Croke, J., Vítkovský, J., Hughes, K., Campbell, M.,
Amirnezhad-Mozhdehi, S., Parnell, A., Cahill, N., \& Dalla Pozza, R.
(2021). A palaeoclimate proxy database for water security planning in
Queensland Australia. \emph{Scientific Data}, \emph{8}(1), 292.
\url{https://doi.org/10.1038/s41597-021-01074-8}}

\leavevmode\hypertarget{ref-mcgowan_evidence_2020}{}%
\CSLLeftMargin{2. }
\CSLRightInline{McGowan, H., Campbell, M., Callow, J. N., Lowry, A., \&
Wong, H. (2020). Evidence of wet-dry cycles and mega-droughts in the
Eemian climate of southeast Australia. \emph{Scientific Reports},
\emph{10}(1), 1--10. \url{https://doi.org/10.1038/s41598-020-75071-z}}

\leavevmode\hypertarget{ref-McGowan2018}{}%
\CSLLeftMargin{3. }
\CSLRightInline{McGowan, H., Callow, J. N., Soderholm, J., McGrath, G.,
Campbell, M., \& Zhao, J. (2018). Global warming in the context of 2000
years of Australian alpine temperature and snow cover. \emph{Scientific
Reports}, \emph{8}(1), 4394.
\url{https://doi.org/10.1038/s41598-018-22766-z}}

\leavevmode\hypertarget{ref-Campbell2018}{}%
\CSLLeftMargin{4. }
\CSLRightInline{Campbell, M., Callow, J., McGrath, G., \& McGowan, H.
(2018). Co-authorship analysis of the speleothem proxy-climate
community: working together to tackle the big problems.
\emph{International Journal of Speleology}, \emph{47}(2), 165--172.
\url{https://doi.org/10.5038/1827-806X.47.2.2159}}

\leavevmode\hypertarget{ref-Campbell2017}{}%
\CSLLeftMargin{5. }
\CSLRightInline{Campbell, M., Callow, J. N., McGrath, G., \& McGowan, H.
(2017). A multimethod approach to inform epikarst drip discharge
modelling: Implications for palaeo-climate reconstruction.
\emph{Hydrological Processes}, \emph{31}(26), 4734--4747.
\url{https://doi.org/10.1002/hyp.11392}}

\hypertarget{conference-presentations-presenting-only}{%
\section{Conference Presentations (presenting
only)}\label{conference-presentations-presenting-only}}

\hypertarget{talks}{%
\subsection{Talks}\label{talks}}

\nopagebreak
    \cventry{Feb 2022}{Reconstructing Australia's fire history from cave stalagmites}{13th International Conference on Southern Hemisphere Meteorology and Oceanography, New Zealand (online)}{Campbell, M., McDonough, L., Kosarac, N., Treble, P., Baker, A}{}{\empty}
    \cventry{Nov 2021}{Applying an open access palaeoclimate database for Australia, climate reconstructions, and water security planning in Queensland}{Seminar Panel, Australia}{Croke, J., Vitkovsky, J., Hughes, K., Campbell, M., Dalla Pozza, R}{}{\empty}
    \cventry{Feb 2017}{A geophysical approach to inform epikarst drip discharge modelling: Implications for palaeo-climate reconstruction}{Australian and New Zealand Geomorphological Group 17th Biennial Conference, Greytown, New Zealand}{Campbell, M., Callow, J.N., McGrath, G.S., McGowan, H.A}{}{\empty}

\hypertarget{posters}{%
\subsection{Posters}\label{posters}}

\nopagebreak
    \cventry{Dec 2021}{Speleothems as palaeofire archives - a synthesis and meta-analysis of data and methods}{American Geophysical Union, Fall Meeting, New Orleans, USA (online)}{Campbell, M., Treble, P., McDonough, L., Baker, A., Kosarac, N}{}{\empty}
    \cventry{Nov 2021}{Speleothem 14C is unlikely to be impacted by wildfire - case studies from Western Australia and Tasmania}{Accelerator Mass Spectroscopy 15th International Conference, Sydney, Australia}{Campbell, M., McDonough, L., Kosarac, N., Treble, P., Markowska, M., Baker, A., Hua, Q}{}{\empty}
    \cventry{Apr 2018}{Climate patterns in South-east Australia: the Last Interglacial vs. the last 2K}{20th EGU General Assembly, Vienna, Austria}{Campbell, M., Callow, J.N., McGowan, H.A., McGrath, G.S., Wong, H}{}{\empty}
    \cventry{Feb 2017}{First Insights of the Eemian Hydroclimate of the Snowy Mountains, Australia}{Australian and New Zealand Geomorphological Group 17th Biennial Conference, Greytown, New Zealand}{Campbell, M., Wong, H., McGrath, G.S., McGowan, H.A., Callow, J.N}{}{\empty}
    \cventry{Dec 2016}{First Insights of the Eemian Hydroclimate of the Snowy Mountains, Australia}{American Geophysical Union, Fall Meeting, San Francisco, USA}{Campbell, M. Wong, H., McGrath, G.S., McGowan, H.A., Callow, J.N}{}{\empty}

\hypertarget{skills}{%
\section{Skills}\label{skills}}

\hypertarget{statistical}{%
\subsection{Statistical}\label{statistical}}

\nopagebreak
    \cventry{}{Statistical Software}{}{}{}{\begin{itemize}%
\item R (advanced)%
\end{itemize}}

\nopagebreak
    \cventry{}{Statistical Methods}{}{}{}{\begin{itemize}%
\item Time series analysis%
\item Hypothesis testing%
\item Social network analysis%
\item Linear models%
\item Generalised linear models%
\item Numerical modelling%
\end{itemize}}

\hypertarget{field}{%
\subsection{Field}\label{field}}

\nopagebreak
    \cventry{}{Surveying}{}{}{}{\begin{itemize}%
\item Total station%
\item Real time kinematic%
\end{itemize}}
    \cventry{}{Electrical Resistivity Tomography}{}{}{}{\begin{itemize}%
\item Deployment%
\item Data processing and analysis%
\end{itemize}}
    \cventry{}{Environmental Data Loggers}{}{}{}{\begin{itemize}%
\item Onset HOBO%
\item In-Situ Rugged Troll%
\item Driptych Stalagmate%
\end{itemize}}
    \cventry{}{Environmental Sampling}{}{}{}{\begin{itemize}%
\item Isotopes in water%
\item Nutrients in water%
\item Soil profiling%
\end{itemize}}

\hypertarget{other}{%
\subsection{Other}\label{other}}

\nopagebreak
    \cventry{}{Software}{}{}{}{\begin{itemize}%
\item Python (fundamental awareness)%
\item ArcGIS (novice)%
\item Microsoft Suite (advanced)%
\item LaTEX (novice)%
\item Rmarkdown (intermediate)%
\end{itemize}}

\hypertarget{service}{%
\section{Service}\label{service}}

\nopagebreak
    \cventry{Aug 2021 -
Present}{Executive Committee Member, Australasian Quaternary Association}{}{}{}{\empty}
    \cventry{Aug 2017 -
Aug 2017}{Participant, Science in Schools Program}{}{}{}{\begin{itemize}%
\item Travelled to a regional town to share some love for science with primary school kids%
\end{itemize}}
    \cventry{Jan 2016 -
Mar 2016}{Co-Convenor, SEE LearnR Workshop}{}{}{}{\empty}
    \cventry{Feb 2015 -
Jul 2017}{Postgraduate representative to the School of Earth and Environment}{}{}{}{\begin{itemize}%
\item Led the organisation of 2 annual symposia for postgraduate students within the school.%
\item Reported student concerns to the School administration%
\item Support and troubleshooting for new students%
\end{itemize}}
    \cventry{Oct 2014 -
Oct 2014}{Guest presenter, ENVT2221: Global Climate Change and Biodiversity}{}{}{}{\begin{itemize}%
\item Presented 'Palaeo-climate and Stalagmites' for 2nd-year Biology students at UWA%
\end{itemize}}

\hypertarget{completed-peer-reviews}{%
\section{Completed Peer Reviews}\label{completed-peer-reviews}}

\nopagebreak
    \cvitem{2021}{Geology. }

\hypertarget{professional-development}{%
\section{Professional Development}\label{professional-development}}

\nopagebreak
    \cvitem{2022}{Science Meets Parliament. Online}
    \cvitem{2020}{Advanced R Workshop. The University of Queensland, Australia}
    \cvitem{2020}{Intermediate R Workshop. The University of Queensland, Australia}
    \cvitem{2018}{Weathering Climate Change: How have humans coped with climate change-and how will we continue to do so. The University of Western Australia, Australia}
    \cvitem{2018}{Masterclass in Creating Opportunities and Developing your Research Skills. The University of Western Australia, Australia}
    \cvitem{2016}{Synthesis II - From dynamics of structure to function of complex networks. TU Dresden, Germany}
    \cvitem{2016}{Water on Earth: origin, reservoirs, and its global cycle. The University of Western Australia, Australia}
    \cvitem{2016}{Fundamentals of the analysis of networks. The University of Western Australia, Australia}
    \cvitem{2014}{Writing and publishing in scientific journals. The University of Western Australia, Australia}

\hypertarget{professional-memberships}{%
\section{Professional Memberships}\label{professional-memberships}}

\nopagebreak
    \cvitem{Jul 2021 -
Present}{Australasian Quaternary Association. }
    \cvitem{Jul 2021 -
Present}{American Geophysical Union. }

\hypertarget{referees}{%
\section{Referees}\label{referees}}

\nopagebreak


\end{document}

%\clearpage\end{CJK*}                              % if you are typesetting your resume in Chinese using CJK; the \clearpage is required for fancyhdr to work correctly with CJK, though it kills the page numbering by making \lastpage undefined
\end{document}


%% end of file `template.tex'.
