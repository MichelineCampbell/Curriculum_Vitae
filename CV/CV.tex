%% start of file `template.tex'.
%% Copyright 2006-2015 Xavier Danaux (xdanaux@gmail.com).
%
% Adapted to be an Rmarkdown template by Mitchell O'Hara-Wild
% 8 February 2019
%
% This work may be distributed and/or modified under the
% conditions of the LaTeX Project Public License version 1.3c,
% available at http://www.latex-project.org/lppl/.


\documentclass[11pt,a4paper,]{moderncv}

% moderncv themes
\moderncvstyle{classic}                             % style options are 'casual' (default), 'classic', 'banking', 'oldstyle' and 'fancy'

\definecolor{color0}{rgb}{0,0,0}% black
\definecolor{color1}{HTML}{125656}% custom
\definecolor{color2}{rgb}{0.45,0.45,0.45}% dark grey

\usepackage[scaled=0.86]{DejaVuSansMono}

\providecommand{\tightlist}{%
	\setlength{\itemsep}{0pt}\setlength{\parskip}{0pt}}
\def\donothing#1{#1}
\def\emaillink#1{#1}

%\nopagenumbers{}                                  % uncomment to suppress automatic page numbering for CVs longer than one page

% character encoding
%\usepackage[utf8]{inputenc}                       % if you are not using xelatex ou lualatex, replace by the encoding you are using
%\usepackage{CJKutf8}                              % if you need to use CJK to typeset your resume in Chinese, Japanese or Korean

% adjust the page margins
\usepackage[scale=0.75,footskip=60pt]{geometry}
%\setlength{\hintscolumnwidth}{3cm}                % if you want to change the width of the column with the dates
%\setlength{\makecvheadnamewidth}{10cm}            % for the 'classic' style, if you want to force the width allocated to your name and avoid line breaks. be careful though, the length is normally calculated to avoid any overlap with your personal info; use this at your own typographical risks...



% personal data
\name{}{Dr.~Micheline Campbell}
\title{Research Fellow, Max Planck Institute for Chemistry}
\address{Mainz, Germany}{}{}

\phone[mobile]{+49 176 36125922} % Phone number
\email{\donothing{\href{mailto:micheline.campbell@mpic.de}{\nolinkurl{micheline.campbell@mpic.de}}}}
\homepage{www.mpic.de/person/136604/3669258} % Personal website


\social[github]{MichelineCampbell}
\social[orcid]{0000-0002-9626-1189}


% \extrainfo{additional information}                 % optional, remove / comment the line if not wanted




% Pandoc CSL macros
\newlength{\cslhangindent}
\setlength{\cslhangindent}{1.5em}
\newlength{\csllabelwidth}
\setlength{\csllabelwidth}{3em}
\newenvironment{CSLReferences}[3] % #1 hanging-ident, #2 entry spacing
 {% don't indent paragraphs
  \setlength{\parindent}{0pt}
  % turn on hanging indent if param 1 is 1
  \ifodd #1 \everypar{\setlength{\hangindent}{\cslhangindent}}\ignorespaces\fi
  % set entry spacing
  \ifnum #2 > 0
  \setlength{\parskip}{#2\baselineskip}
  \fi
 }%
 {}
\usepackage{calc}
\newcommand{\CSLBlock}[1]{#1\hfill\break}
\newcommand{\CSLLeftMargin}[1]{\parbox[t]{\csllabelwidth}{#1}}
\newcommand{\CSLRightInline}[1]{\parbox[t]{\linewidth - \csllabelwidth}{#1}}
\newcommand{\CSLIndent}[1]{\hspace{\cslhangindent}#1}

%----------------------------------------------------------------------------------
%            content
%----------------------------------------------------------------------------------
\begin{document}
%\begin{CJK*}{UTF8}{gbsn}                          % to typeset your resume in Chinese using CJK
%-----       resume       ---------------------------------------------------------
\makecvtitle



\hypertarget{professional-summary}{%
\section{Professional summary}\label{professional-summary}}

My research is at the intersection of earth and environmental science. I
use proxy data from natural archives to investigate past environments,
requiring research into the modern environment and how that affects the
proxy system. I have published on a wide array of topics, and have
experience supervising both Honours and PhD students. I have had funding
success in both research grants and access to facilities. I aim to
secure a position which will allow me to apply my laboratory and data
science skillset to answer some of the world's wicked questions and help
policy-makers and land managers plan for a changing world. Beyond my own
research, I plan to find a position where I can continue to support and
empower the next generation of earth and environmental scientists.

\hypertarget{education}{%
\section{Education}\label{education}}

\nopagebreak
    \cventry{Aug 2014 -
Jul 2019}{PhD (Geography)}{School of Agriculture and Environment, The University of Western Australia}{Perth, Western Australia, Australia}{}{\begin{itemize}%
\item  \textbf{Thesis: Speleothem-based palaeo-climate research: Methodology, applications, and insight from the Snowy Mountains, southeast Australia}%
\item Supervisors: JN. Callow, H. McGowan, G.McGrath, H. Wong%
\end{itemize}}
    \cventry{Jul 2013 -
Jun 2014}{BA (Hons, 1st Class) (Geography)}{School of Geography, Planning, and Environmental Management, The University of Queensland}{Brisbane, QLD, Australia}{}{\begin{itemize}%
\item  \textbf{Thesis: The Drip Hydrology of a Speleothem from the Yarrangobilly Caves, NSW, Australia}%
\item Supervisors: H. Mcgowan, JN. Callow%
\end{itemize}}\nopagebreak
    \cventry{Feb 2010 -
Dec 2012}{BA (Geography and Spanish)}{The University of Queensland}{Brisbane, QLD, Australia}{}{\empty}

\hypertarget{experience}{%
\section{Experience}\label{experience}}

\hypertarget{research}{%
\subsection{Research}\label{research}}

\nopagebreak
    \cventry{Aug 2024 -
present}{Research Fellow}{Department of Climate Geochemistry, Max Planck Institute for Chemistry}{Mainz, Rhineland-Palatinate, Germany}{}{\begin{itemize}%
\item Reconstructing past fire frequencies from stalagmites.%
\end{itemize}}
    \cventry{Aug 2024 -
present}{Adjunct Fellow}{School of Biological, Earth and Environmental Sciences, University of New South Wales}{Sydney, NSW, Australia}{}{\empty}
    \cventry{Jun 2021 -
Jun 2024}{Research Associate}{School of Biological, Earth and Environmental Sciences, University of New South Wales}{Sydney, NSW, Australia}{}{\begin{itemize}%
\item Contributed to the ARC Discovery Project ``Reconstructing Australia's fire history from cave stalagmites''%
\item Position was jointly appointed with UNSW and ANSTO%
\end{itemize}}
    \cventry{Jul 2020 -
Jun 2021}{Postdoctoral Researcher}{School of Geography, University College Dublin}{Dublin/Brisbane, QLD, Ireland/Australia}{}{\begin{itemize}%
\item Contributed to the project 'Using Palaeo-climate Proxies for Water Security Planning'%
\item I undertook the majority of data science tasks for the development of a new database of Australian climate proxy data%
\item During this position I was hosted by the Queensland Department of Environment and science, where I worked closely with DES hydrologists and liaised with other government stakeholders (e.g. SEQWater)%
\end{itemize}}
    \cventry{Feb 2020 -
Jul 2020}{Scientist}{Land and Water Science Unit, QLD Dept. of Natural Resources, Mining and Energy}{Rockhampton, QLD, Australia}{}{\empty}

\hypertarget{teaching}{%
\subsection{Teaching}\label{teaching}}

Teaching contributions include co-coordinating, lecturing, and tutoring
at both the undergraduate and postgraduate levels. This has included the
creation of original content, management of fieldwork, content revision,
and grading. \(~\)

\(~\)

\nopagebreak
    \cventry{2024}{Guest presenter: Earth and Environmental Science}{School of Biological, Earth and Environmental Sciences, UNSW Sydney}{Sydney, NSW, Australia}{}{\empty}
    \cventry{2018}{Tutor: Geographies of a Global City}{School of Agriculture and Enviroment, The University of Western Australia}{Perth, WA, Australia}{}{\empty}
    \cventry{2018}{Course Co-Coordinator and Lecturer: Catchment and River Processes}{School of Agriculture and Enviroment, The University of Western Australia}{Perth, WA, Australia}{}{\empty}
    \cventry{2014--2018}{Tutor: Catchment and River Processes}{School of Agriculture and Enviroment, The University of Western Australia}{Perth, WA, Australia}{}{\empty}

\hypertarget{grants-and-awards}{%
\section{Grants and Awards}\label{grants-and-awards}}

Total funding to date exceeds AUD\$800 000, with \textgreater AUD\$240
000 of that awarded as the lead investigator. Key funders include the
Australian Institute of Nuclear Science and Engineering (two research
awards), ANSTO (six synchrotron access proposals), and UNSW (two
competitive internal funding schemes, equivalent to \textasciitilde3
months salary and research costs). \(~\)

\(~\)

\nopagebreak
    \cventry{2024}{ANSTO 96  hours of beamtime on the XFM line at the Australian Synchrotron (experiment 21875)}{}{Investigator, in-kind support}{}{\begin{itemize}%
\item \$131 136%
\end{itemize}}
    \cventry{2024}{ANSTO 72 hours of beamtime on the XFM line at the Australian Synchrotron (experiment 23245)}{}{Investigator, in-kind support}{}{\begin{itemize}%
\item \$98 352%
\end{itemize}}
    \cventry{2023}{ANSTO 24  hours of beamtime on the XFM line at the Australian Synchrotron (experiment 19157)}{}{Lead investigator, in-kind support}{}{\begin{itemize}%
\item \$32 784%
\end{itemize}}
    \cventry{2022}{ANSTO 48 hours of beamtime on the XFM line at the Australian Synchrotron (experiment 17897)}{}{Investigator, in-kind support}{}{\begin{itemize}%
\item \$65 568%
\end{itemize}}
    \cventry{2022}{ANSTO 144 hours of beamtime on the IRM line at the Australian Synchrotron (experiment 17905)}{}{Investigator, in-kind support}{}{\begin{itemize}%
\item \$196 704%
\end{itemize}}
    \cventry{2022}{ANSTO 48 hours of beamtime on the IRM line at the Australian Synchrotron (experiment 18777)}{}{Investigator, in-kind support}{}{\begin{itemize}%
\item \$65 568%
\end{itemize}}
    \cventry{2021}{Australian Institute of Nuclear Science and Engineering Early Career Researcher Grant}{}{Lead investigator}{}{\begin{itemize}%
\item \$10 000%
\end{itemize}}
    \cventry{2021}{University of New South Wales UNSW Science Covid-19 Strategic Support Grant}{}{Lead investigator}{}{\begin{itemize}%
\item \$4 000%
\end{itemize}}
    \cventry{2021}{UNSW Research Stopgap Scheme}{}{Lead investigator, salary costs. Competitive funding internal to UNSW}{}{\begin{itemize}%
\item \$39 766%
\end{itemize}}
    \cventry{2018}{Australian Institute of Nuclear Science and Engineering Travel grant}{}{Lead investigator}{}{\begin{itemize}%
\item \$1 000%
\end{itemize}}
    \cventry{2017}{Australian and New Zealand Geomorphological Group Travel grant}{}{Lead investigator}{}{\begin{itemize}%
\item \$300%
\end{itemize}}
    \cventry{2016}{University of Western Australia Travel grant}{}{Lead investigator}{}{\begin{itemize}%
\item \$1 850%
\end{itemize}}
    \cventry{2015--2018}{Australian Institute of Nuclear Science and Engineering Postgraduate Ressarch Award Stipend}{}{Lead investigator}{}{\begin{itemize}%
\item \$22 500%
\end{itemize}}
    \cventry{2015--2018}{Australian Institute of Nuclear Science and Engineering Postgraduate Research Award Analytical Budget}{}{Lead investigator, in-kind support}{}{\begin{itemize}%
\item \$30 000%
\end{itemize}}
    \cventry{2014--2018}{Australian Government Research Training Program Stipend}{}{Lead investigator}{}{\begin{itemize}%
\item \$91 000%
\end{itemize}}
    \cventry{2014--2015}{Snowy Hydro Ltd. Stipend}{}{Lead}{}{\begin{itemize}%
\item \$10 000%
\end{itemize}}
    \cventry{2014}{University of Western Australia Stipend}{}{Lead}{}{\begin{itemize}%
\item \$3 000%
\end{itemize}}

\hypertarget{publications}{%
\section{Publications}\label{publications}}

\hypertarget{bibliography}{}
\leavevmode\vadjust pre{\hypertarget{ref-song_rainfall_2025}{}}%
\CSLLeftMargin{1. }%
\CSLRightInline{Song, C., Campbell, M., \& Baker, A. (2025). Rainfall
recharge thresholds decrease after an intense fire over a near-surface
cave at Wombeyan, Australia. \emph{Hydrology and Earth System Sciences},
\emph{29}(17), 4241--4250.
\url{https://doi.org/10.5194/hess-29-4241-2025}}

\leavevmode\vadjust pre{\hypertarget{ref-mcdonough_fire-induced_2024}{}}%
\CSLLeftMargin{2. }%
\CSLRightInline{McDonough, L. K., Campbell, M., Treble, P. C., Marjo,
C., Frisia, S., Vongsvivut, J., Klein, A. R., Kovacs-Kis, V., \& Baker,
A. (2024). Fire-induced shifts in stalagmite organic matter mapped using
Synchrotron infrared microspectroscopy. \emph{Organic Geochemistry},
\emph{195}, 104842.
\url{https://doi.org/10.1016/j.orggeochem.2024.104842}}

\leavevmode\vadjust pre{\hypertarget{ref-campbell_combustion_2024}{}}%
\CSLLeftMargin{3. }%
\CSLRightInline{Campbell, M., Treble, P. C., McDonough, L. K., Naeher,
S., Baker, A., Grierson, P. F., Wong, H., \& Andersen, M. S. (2024).
Combustion completeness and sample location determine wildfire ash
leachate chemistry. \emph{Geochemistry, Geophysics, Geosystems},
\emph{25}, e2024GC011470. \url{https://doi.org/10.1029/2024GC011470}}

\leavevmode\vadjust pre{\hypertarget{ref-campbell_review_2023}{}}%
\CSLLeftMargin{4. }%
\CSLRightInline{Campbell, M., McDonough, L., Treble, P. C., Baker, A.,
Kosarac, N., Coleborn, K., Wynn, P. M., \& Schmitt, A. K. (2023). A
Review of Speleothems as Archives for Paleofire Proxies, With Australian
Case Studies. \emph{Reviews of Geophysics}, \emph{61}(2), e2022RG000790.
\url{https://doi.org/10.1029/2022RG000790}}

\leavevmode\vadjust pre{\hypertarget{ref-cahill_bayesian_nodate}{}}%
\CSLLeftMargin{5. }%
\CSLRightInline{Cahill, N., Croke, J., Campbell, M., Hughes, K.,
Vitkovsky, J., Kilgallen, J. E., \& Parnell, A. (2023). A Bayesian time
series model for reconstructing hydroclimate from multiple proxies.
\emph{Environmetrics}, \emph{n/a}(n/a), e2786.
\url{https://doi.org/10.1002/env.2786}}

\leavevmode\vadjust pre{\hypertarget{ref-croke_palaeoclimate_2021}{}}%
\CSLLeftMargin{6. }%
\CSLRightInline{Croke, J., Vítkovský, J., Hughes, K., Campbell, M.,
Amirnezhad-Mozhdehi, S., Parnell, A., Cahill, N., \& Dalla Pozza, R.
(2021). A palaeoclimate proxy database for water security planning in
Queensland Australia. \emph{Scientific Data}, \emph{8}(1), 292.
\url{https://doi.org/10.1038/s41597-021-01074-8}}

\leavevmode\vadjust pre{\hypertarget{ref-mcgowan_evidence_2020}{}}%
\CSLLeftMargin{7. }%
\CSLRightInline{McGowan, H., Campbell, M., Callow, J. N., Lowry, A., \&
Wong, H. (2020). Evidence of wet-dry cycles and mega-droughts in the
Eemian climate of southeast Australia. \emph{Scientific Reports},
\emph{10}(1), 1--10. \url{https://doi.org/10.1038/s41598-020-75071-z}}

\leavevmode\vadjust pre{\hypertarget{ref-McGowan2018}{}}%
\CSLLeftMargin{8. }%
\CSLRightInline{McGowan, H., Callow, J. N., Soderholm, J., McGrath, G.,
Campbell, M., \& Zhao, J. (2018). Global warming in the context of 2000
years of Australian alpine temperature and snow cover. \emph{Scientific
Reports}, \emph{8}(1), 4394.
\url{https://doi.org/10.1038/s41598-018-22766-z}}

\leavevmode\vadjust pre{\hypertarget{ref-Campbell2018}{}}%
\CSLLeftMargin{9. }%
\CSLRightInline{Campbell, M., Callow, J., McGrath, G., \& McGowan, H.
(2018). Co-authorship analysis of the speleothem proxy-climate
community: working together to tackle the big problems.
\emph{International Journal of Speleology}, \emph{47}(2), 165--172.
\url{https://doi.org/10.5038/1827-806X.47.2.2159}}

\leavevmode\vadjust pre{\hypertarget{ref-Campbell2017}{}}%
\CSLLeftMargin{10. }%
\CSLRightInline{Campbell, M., Callow, J. N., McGrath, G., \& McGowan, H.
(2017). A multimethod approach to inform epikarst drip discharge
modelling: Implications for palaeo-climate reconstruction.
\emph{Hydrological Processes}, \emph{31}(26), 4734--4747.
\url{https://doi.org/10.1002/hyp.11392}}

\newpage

\hypertarget{media-and-reports}{%
\section{Media and reports}\label{media-and-reports}}

\hypertarget{bibliography}{}
\leavevmode\vadjust pre{\hypertarget{ref-campbell_AQUAMentoring2}{}}%
\CSLLeftMargin{1. }%
\CSLRightInline{Campbell, M., \& Mather, C. (2025). AQUA's mentoring
program - four years in! In \emph{Quaternary Australasia} (Vol. 42).}

\leavevmode\vadjust pre{\hypertarget{ref-campbell_AINSE}{}}%
\CSLLeftMargin{2. }%
\CSLRightInline{Campbell, M., Treble, P., McDonough, L., Baker, A.,
Wong, H., \& Hankin, S. (2024). Cave speleothems record past wildfires.
In \emph{AINSE Annual Report 2023}.}

\leavevmode\vadjust pre{\hypertarget{ref-campbell_using_2023}{}}%
\CSLLeftMargin{3. }%
\CSLRightInline{Campbell, M., McDonough, L., Treble, P. C., \& Baker, A.
(2023). Using Cave Formations to Investigate Ancient Wildfires.
\emph{Eos}, \emph{104}.
\url{http://eos.org/editors-vox/using-cave-formations-to-investigate-ancient-wildfires}}

\leavevmode\vadjust pre{\hypertarget{ref-campbell_AQUAgender}{}}%
\CSLLeftMargin{4. }%
\CSLRightInline{Campbell, M., \& Barrows, T. (2023). Continuing the
conversation on equity, diversity, and inclusion in AQUA. In
\emph{Quaternary Australasia} (Vol. 40).}

\leavevmode\vadjust pre{\hypertarget{ref-campbell_AQUAFunding}{}}%
\CSLLeftMargin{5. }%
\CSLRightInline{Committee, A. E. (2023). Open letter on the future of
geosciences in higher education. In \emph{Quaternary Australasia} (Vol.
40).}

\leavevmode\vadjust pre{\hypertarget{ref-campbell_AQUAMentoring1}{}}%
\CSLLeftMargin{6. }%
\CSLRightInline{Campbell, M., \& Mather, C. (2022). The inaugural AQUA
mentoring program. In \emph{Quaternary Australasia} (Vol. 39).}

\hypertarget{presentations-presenting-only}{%
\section{Presentations (presenting
only)}\label{presentations-presenting-only}}

\hypertarget{talks}{%
\subsection{Talks}\label{talks}}

\nopagebreak
    \cventry{Mar 2025}{Speleothem inorganic elements as high-resolution proxies of past fire}{Climate Change, The Karst Record X, Cape Town, South Africa}{Campbell, M., Treble, P.C., McDonough, L., Markowska, M., Vonhof, H., Baker, A}{}{\empty}
    \cventry{Dec 2022}{Towards the development of fire proxies in speleothems using geochemical signatures in ashes from bushfires}{2022 AQUA Biennial, Adelaide, Australia}{Campbell, M., McDonough, L., Naeher, S., Treble, P., Grierson, P., Sinclair, D., Howard, D., Baker, A}{}{\empty}
    \cventry{Feb 2022}{Reconstructing Australia's fire history from cave stalagmites}{13th International Conference on Southern Hemisphere Meteorology and Oceanography, New Zealand (online)}{Campbell, M., McDonough, L., Kosarac, N., Treble, P., Baker, A}{}{\empty}
    \cventry{Feb 2017}{A geophysical approach to inform epikarst drip discharge modelling: Implications for palaeo-climate reconstruction}{Australian and New Zealand Geomorphological Group 17th Biennial Conference, Greytown, New Zealand}{Campbell, M., Callow, J.N., McGrath, G.S., McGowan, H.A}{}{\empty}

\hypertarget{posters}{%
\subsection{Posters}\label{posters}}

\nopagebreak
    \cventry{Jun 2024}{Exploring a speleothem fire severity signal using coeval stalagmites from southwest Australia}{2024 AQUA Biemmial, Minjerribah, Australia}{Campbell, M., Treble, P.C., Baker, A., McDonough, L., Howard, D., Sinclair, D}{}{\empty}
    \cventry{Apr 2023}{Speleothems as archives for palaeofire proxies}{EGU General Assembly, Vienna, Austria}{Campbell, M., McDonough, L., Treble, P.C., Baker, A., Kosarac, N., Coleborn, K., Wynn, P., Schmitt, A.K}{}{\empty}
    \cventry{Dec 2022}{Stalagmites as high-resolution archives of past fire severity}{Synchrotron User Meeting, Melbourne, Australia}{Campbell, M., McDonough, L., Treble, P., Baker, A., Howard, D.,Hankin, S}{}{\empty}
    \cventry{Jul 2022}{Stalagmites as high-resolution archives of past fire severity}{Climate Change, The Karst Record IX, Innsbruck, Austria}{Campbell, M., McDonough, L., Treble, P., Baker, A., Howard, D.,Hankin, S}{}{\empty}
    \cventry{Dec 2021}{Speleothems as palaeofire archives - a synthesis and meta-analysis of data and methods}{American Geophysical Union, Fall Meeting, New Orleans, USA (online)}{Campbell, M., Treble, P., McDonough, L., Baker, A., Kosarac, N}{}{\empty}
    \cventry{Nov 2021}{Speleothem 14C is unlikely to be impacted by wildfire - case studies from Western Australia and Tasmania}{Accelerator Mass Spectroscopy 15th International Conference, Sydney, Australia}{Campbell, M., McDonough, L., Kosarac, N., Treble, P., Markowska, M., Baker, A., Hua, Q}{}{\empty}
    \cventry{Apr 2018}{Climate patterns in South-east Australia: the Last Interglacial vs. the last 2K}{EGU General Assembly, Vienna, Austria}{Campbell, M., Callow, J.N., McGowan, H.A., McGrath, G.S., Wong, H}{}{\empty}
    \cventry{Feb 2017}{First Insights of the Eemian Hydroclimate of the Snowy Mountains, Australia}{Australian and New Zealand Geomorphological Group 17th Biennial Conference, Greytown, New Zealand}{Campbell, M., Wong, H., McGrath, G.S., McGowan, H.A., Callow, J.N}{}{\empty}
    \cventry{Dec 2016}{First Insights of the Eemian Hydroclimate of the Snowy Mountains, Australia}{American Geophysical Union, Fall Meeting, San Francisco, USA}{Campbell, M. Wong, H., McGrath, G.S., McGowan, H.A., Callow, J.N}{}{\empty}

\hypertarget{invited-talks-and-seminars}{%
\subsection{Invited talks and
seminars}\label{invited-talks-and-seminars}}

\nopagebreak
    \cventry{Aug 2025}{Fire signals from Australian speleothems}{ICCP speleothem lecture series, Busan, Korea}{Campbell, M}{}{\empty}
    \cventry{Aug 2025}{Fire signals from Australian speleothems}{Northumberland University, Newcastle, UK}{Campbell, M}{}{\empty}
    \cventry{Jul 2025}{ECR Panel}{AINSE Winter School, Sydney, Australia}{Campbell, M}{}{\empty}
    \cventry{Nov 2024}{Reconstructing past fires from stalagmites - the inorganic proxy record}{Heidelberg University Institute of Geosciences Seminar Series, Heidelberg, Germany}{Campbell, M}{}{\empty}
    \cventry{Nov 2023}{Australasian palaeoclimate - developing databases and novel proxy applications}{UOW Environmental Futures Seminar Series, Wollongong, Australia}{Campbell, M}{}{\empty}
    \cventry{Jun 2023}{Towards the development of fire proxies in speleothems using geochemical signatures in ashes from bushfires}{UTAS IMAS Seminar Series, Hobart, Australia}{Campbell, M}{}{\empty}
    \cventry{Nov 2022}{Palaeoclimate, Palaeofire, and Palaeodatabases}{UNSW BEES Seminar Series, Sydney, Australia}{Campbell, M}{}{\empty}
    \cventry{Apr 2022}{Stalagmites as high-resolution archives of past fire severity}{ANSTO Science Talks, Sydney, Australia}{Campbell, M}{}{\empty}
    \cventry{Nov 2021}{Applying an open access palaeoclimate database for Australia, climate reconstructions, and water security planning in Queensland}{NA, Australia}{Croke, J., Vitkovsky, J., Hughes, K., Campbell, M., Dalla Pozza, R}{}{\empty}

\hypertarget{skills}{%
\section{Skills}\label{skills}}

\hypertarget{statistical}{%
\subsection{Statistical}\label{statistical}}

\nopagebreak

\nopagebreak
    \cventry{}{Statistical Methods}{}{}{}{\begin{itemize}%
\item Time series analysis%
\item Hypothesis testing%
\item Network analysis%
\item Linear models%
\item Numerical hydrological modelling%
\item Principal Component Analysis and other dimension reduction techniques%
\item Data wrangling%
\item GIS (R suite of spatial analysis tools; mapping; manipulation and analysis of gridded data; calculation of spatial statistics)%
\item Remote sensing (dNBR; MODIS Burned area)%
\end{itemize}}

\hypertarget{field}{%
\subsection{Field}\label{field}}

\nopagebreak
    \cventry{}{Surveying}{}{}{}{\begin{itemize}%
\item Total station%
\item Real time kinematic%
\end{itemize}}
    \cventry{}{Environmental Monitoring}{}{}{}{\begin{itemize}%
\item Deployment and maintenance of environmental logger networks (e.g. discharge and temperature, pressure, and RH loggers)%
\end{itemize}}
    \cventry{}{Environmental Sampling}{}{}{}{\begin{itemize}%
\item Stable isotopes in water%
\item Nutrients in water%
\end{itemize}}

\hypertarget{laboratory}{%
\subsection{Laboratory}\label{laboratory}}

\nopagebreak
    \cventry{}{Sample preparation}{}{}{}{\begin{itemize}%
\item Milling carbonates for IRMS%
\item Thick section mounting for LA-ICP-MS and Synchrotron XFM%
\end{itemize}}
    \cventry{}{Analysis}{}{}{}{\begin{itemize}%
\item LA-ICP-MS and data reduction to quantify trace elements in speleothems%
\item Synchrotron XFM elemental mapping and data processing to quantify trace elements in speleothems%
\end{itemize}}

\hypertarget{other}{%
\subsection{Other}\label{other}}

\nopagebreak
    \cventry{}{Software}{}{}{}{\begin{itemize}%
\item R (advanced)%
\item Python (fundamental awareness)%
\item QGIS (intermediate)%
\item Microsoft Suite (advanced)%
\item LaTEX (novice)%
\item Rmarkdown (advanced)%
\end{itemize}}

\hypertarget{service}{%
\section{Service}\label{service}}

\nopagebreak
    \cventry{Aug 2021 -
Present}{Executive Committee Member, Australasian Quaternary Association}{}{}{}{\begin{itemize}%
\item IT and communications officer 2022-2024%
\item Organising committee for AQUA's mentorship program in 2022 and 2024%
\item Student prize and travel award sub-committee%
\item IT support for 2022 and 2024 AQUA Biennial Meetings%
\end{itemize}}
    \cventry{Sep 2024 -
Present}{Organiser - Vonhof Group Journal Club, MPIC}{}{}{}{\empty}
    \cventry{Oct 2024 -
Present}{Organiser - Climate Geochemistry Shut Up and Write, MPIC}{}{}{}{\empty}
    \cventry{May 2025 -
May 2025}{Session Convenor, EGU General Assembly 2025}{}{}{}{\begin{itemize}%
\item Co-Convenor for the session ``Understanding the carbon cycle - climate interactions during the Quaternary through the study of oceanic circulation, vegetation, and wildfire''%
\end{itemize}}
    \cventry{Jul 2023 -
Jul 2023}{Session Convenor, INQUA Roma 2023}{}{}{}{\begin{itemize}%
\item Co-Convenor for the session ``Session 118: Cave deposits for in deep understanding Quaternary climate and environment''%
\end{itemize}}
    \cventry{Apr 2023 -
Apr 2023}{Session Convenor, EGU General Assembly 2023}{}{}{}{\begin{itemize}%
\item Co-Convenor and co-chair for session ``Fire in the Earth system: understanding effects across spatiotemporal scales''%
\end{itemize}}
    \cventry{Nov 2022 -
Nov 2022}{University Of Wollongong Honours thesis examiner}{}{}{}{\empty}
    \cventry{Jul 2022 -
Jul 2022}{Session Co-Chair, Climate Change the Karst Record IX}{}{}{}{\begin{itemize}%
\item Co-Chair for session 'Cave Monitoring'%
\end{itemize}}
    \cventry{Aug 2017 -
Aug 2017}{Participant, Science in Schools Program}{}{}{}{\begin{itemize}%
\item Travelled to a regional town to share some love for science with primary school kids%
\end{itemize}}
    \cventry{Feb 2015 -
Jul 2017}{Postgraduate representative to the School of Earth and Environment}{}{}{}{\begin{itemize}%
\item Led the organisation of 2 annual symposia for postgraduate students within the school.%
\item Reported student concerns to the School administration%
\item Support and troubleshooting for new students%
\end{itemize}}
    \cventry{Jan 2016 -
Mar 2016}{Co-Convenor, SEE LearnR Workshop}{}{}{}{\begin{itemize}%
\item Organised a short series of 'Introduction to R' courses for the students and staff of the School of Earth and Environment%
\item Presented 'Introduction to ggplot2'%
\item Presented 'Introduction to R'%
\end{itemize}}
    \cventry{Oct 2014 -
Oct 2014}{Guest presenter, ENVT2221: Global Climate Change and Biodiversity}{}{}{}{\begin{itemize}%
\item Presented 'Palaeo-climate and Stalagmites' for 2nd-year Biology students at UWA%
\end{itemize}}

\hypertarget{completed-peer-reviews}{%
\section{Completed peer reviews}\label{completed-peer-reviews}}

\nopagebreak
    \cvitem{2024}{Biogeochemistry. }
    \cvitem{2023}{Climate of the Past. }
    \cvitem{2023}{Nature Communications. }
    \cvitem{2023}{Geochimica et cosmochimica acta. }
    \cvitem{2022}{Paleoceanography and Paleoclimatology. }
    \cvitem{2021}{Geology. }

\hypertarget{professional-development}{%
\section{Professional development}\label{professional-development}}

\nopagebreak
    \cvitem{2025}{Speleothem Petrography: a tool to recognize local hydrology. University of Cape Town, South Africa}
    \cvitem{2022}{Science Meets Parliament. Online}
    \cvitem{2020}{Advanced R Workshop. The University of Queensland, Australia}
    \cvitem{2020}{Intermediate R Workshop. The University of Queensland, Australia}
    \cvitem{2018}{Weathering Climate Change: How have humans coped with climate change-and how will we continue to do so. The University of Western Australia, Australia}
    \cvitem{2018}{Masterclass in Creating Opportunities and Developing your Research Skills. The University of Western Australia, Australia}
    \cvitem{2016}{Synthesis II - From dynamics of structure to function of complex networks. TU Dresden, Germany}
    \cvitem{2016}{Water on Earth: origin, reservoirs, and its global cycle. The University of Western Australia, Australia}
    \cvitem{2016}{Fundamentals of the analysis of networks. The University of Western Australia, Australia}
    \cvitem{2014}{Writing and publishing in scientific journals. The University of Western Australia, Australia}

\hypertarget{professional-memberships}{%
\section{Professional memberships}\label{professional-memberships}}

\nopagebreak
    \cvitem{Aug 2021 -
Present}{European Geosciences Union. }
    \cvitem{Jul 2021 -
Present}{Australasian Quaternary Association. }
    \cvitem{Jul 2021 -
Present}{American Geophysical Union. }

\hypertarget{referees}{%
\section{Referees}\label{referees}}

\nopagebreak
    \cventry{Post-Doctorate Supervisor}{Professor Andy Baker}{School of Biological, Earth and Environmental Sciences, The University of New South Wales}{a.baker@unsw.edu.au}{}{\empty}
    \cventry{Post-Doctorate Supervisor}{Dr. Pauline Treble}{Australian Nuclear Science and Technology Organisation}{ptr@ansto.gov.au}{}{\empty}
    \cventry{PhD Supervisor}{Dr. Nik Callow}{School of Agriculture and Environment, The University of Western Australia}{nik.callow@uwa.edu.au; +61 8 6488 1924}{}{\empty}


\end{document}

%\clearpage\end{CJK*}                              % if you are typesetting your resume in Chinese using CJK; the \clearpage is required for fancyhdr to work correctly with CJK, though it kills the page numbering by making \lastpage undefined
\end{document}


%% end of file `template.tex'.
